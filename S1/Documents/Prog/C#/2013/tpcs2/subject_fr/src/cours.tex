\section{Cours}

\subsection{La console}
En bons utilisateurs de Windows vous aimez cliquer sur des boutons avec votre
souris dans votre jolie interface graphique. Mais vous ne savez peut-être pas
qu'il existe un autre monde, plus rapide, plus sombre, qui fera croire à toute
personne extérieure que vous êtes un hacker confirmé: la console.\\
En effet, pendant la préhistoire\footnote{Au moins\ldots 1970 !}, les
informaticiens n'avaient à leur disposition qu'une interface en ligne de
commande dans un écran affichant 80 colonnes de texte.\\
Malgré l'utilisation massive des interfaces graphiques (GUI = Graphical User
Interface) de nos jours, la console et ses commandes textuelles résistent encore
et toujours grâce à leur efficacité.

\subsection{L'invite de commandes Windows}
Chaque version de Windows dispose d'une console héritée de l'époque de MS-DOS
et accessible en lançant \textbf{cmd.exe} (avec Win+R par exemple): l'invite de
commandes.\\
Vous pouvez dans celle-ci exécuter tout un tas d'actions mais aussi les
regrouper dans des scripts afin d'automatiser des tâches
facilement.\footnote{Et oui, un .bat ne sert pas qu'à détruire votre
ordinateur.}\\
Voici les principales commandes:

\begin{description}
    \item[dir] \hfill \\
        Liste les fichiers du dossier courant.
    \item[cd] \hfill \\
        Change le dossier courant.
    \item[copy] \hfill \\
        Copie un fichier à un autre endroit.
    \item[mov] \hfill \\
        Déplace un fichier.
    \item[print] \hfill \\
        Affiche un fichier.
\end{description}

Il est également possible de lancer n'importe quel exécutable simplement en
tapant son nom (précédé du chemin pour y accéder s'il n'est pas dans le dossier
courant).\\
Cette interface est toutefois assez limitée et n'est aujourd'hui conservée que
pour des raisons de compatibilité.

\subsection{Powershell}
Comme son nom l'indique (un shell est un interpréteur de commandes) il s'agit
d'une nouvelle version de la console Windows introduite dans Windows 7.\\
Son fonctionnement se rapproche plus des équivalents Unix avec des noms de
commandes proches et l'apparitions de fonctionnalités évolués telles que les
redirections de flux.
\footnote{http://fr.wikibooks.org/wiki/Programmation\_Bash/Flux\_et\_redirections}\\
Les commandes expliquées précédemment deviennent celles-ci:

\begin{description}
    \item[ls] \hfill \\
        Liste les fichiers du dossier courant.
    \item[cd] \hfill \\
        Change le dossier courant.
    \item[cp] \hfill \\
        Copie un fichier à un autre endroit.
    \item[mv] \hfill \\
        Déplace un fichier.
    \item[cat] \hfill \\
        Affiche un fichier.
\end{description}

On note également l'apparition de \textbf{man} qui permet d'obtenir toutes les
informations nécessaires à l'utilisation d'une commande en tapant man suivi de
celle ci.

\subsection{La console sur Unix}
À partir de l'année prochaine (ou dès maintenant si vous voulez !) vous allez
utiliser des systèmes d'exploitation Unix, dans ces derniers l'interface en
ligne de commandes est encore très présente et est de ce fait très développée
afin de pouvoir n'utiliser qu'elle sans être limité.\\
C'est l'une des raisons qui fait que la majorité des serveurs tourne sous des
OS de ce genre car lors d'un accès à distance seul un
TTY\footnote{http://en.wikipedia.org/wiki/Teleprinter} est accessible.\\
On ne détaillera pas ici toutes les merveilles que vous pouvez y trouver mais
si vous êtes intéressés n'hésitez pas à vous renseigner sur internet où à
demander à vos ACDC.

\subsection{Utilisation en \csharp}
La manipulation de la console en \csharp s'effectue au travers de la
classe\ldots Console !\\
Sa documentation est disponible dans la bible du \csharp: MSDN.
Vous pouvez y accéder ici:
http://msdn.microsoft.com/en-us/library/system.console\\
Votre premier travail dans ce TP sera d'aller rechercher (pas forcément dans
la page donnée au dessus) au moins les fonctions suivantes afin de comprendre
leur utilisation:

\begin{enumerate}
    \item Console.Write
    \item Console.WriteLine
    \item Console.Read
    \item Console.ReadLine
    \item Convert.ToInt32
    \item Convert.ToFloat
\end{enumerate}

\subsection{La fonction Main}
Lorsque vous compilerez et exécuterez votre programme, la première et seule 
fonction appelée sera la fonction \codeil{Main()}.\\
Cette fonction est le point d'entrée de votre programme, c'est donc à l'intérieur
de celle-ci que vous pouvez faire appel à vos autres fonctions afin de les tester.

\newpage
