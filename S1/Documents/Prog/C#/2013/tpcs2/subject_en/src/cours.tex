\section{Course}

\subsection{The console}
As a good Windows user, you like to click on buttons with your mouse.
But you may not know that another world exists -- faster, darker, a world that will make
any average person think that you're a hacker: the console.\\
Indeed, in the old days\footnote{Earlier\ldots{} 1970!}, computer scientists had at their disposal only a command line interface on a screen displaying only 80 columns.\\
Despite the massive use of graphical interfaces (GUI = Graphical User
Interface) nowadays, the console and its text commands still remain thanks to its
efficiency.

\subsection{The Windows terminal}
Every Windows version has a console inherited from MS-DOS
that you can open by running \textbf{cmd.exe} (with Win+R for instance):
the Windows terminal.\\
In this terminal you can execute a whole lot of commands and group them
in files to automate actions easily.\footnote{Yes, a .bat file is not only used to destroy your computer.}\\
Here are the main commands:

\begin{description}
    \item[dir] \hfill \\
        Lists the files in the current directory.
    \item[cd] \hfill \\
        Changes the current directory.
    \item[copy] \hfill \\
        Copies a file to another place.
    \item[mov] \hfill \\
        Moves a file or directory.
    \item[print] \hfill \\
        Prints a file.
\end{description}

You an also run an executable just by typing its name (prefixed by the path to it if
different from the current one). This interface is very limited and outdated and is only kept for compatibility reasons.

\subsection{Powershell}
As its name says --a shell is a command interpreter, it is
a new version of the Windows terminal introduced in Windows 7.\\
Its functionalities are closer to their Unix equivalents with similar command names
and the appearance of advanced functions like stream redirection.\footnote{http://www.tldp.org/LDP/abs/html/io-redirection.html}\\
The previously explained commands become the following:
\newpage
\begin{description}
    \item[ls] \hfill \\
        Lists the files in the current directory.
    \item[cd] \hfill \\
        Changes the current directory.
    \item[cp] \hfill \\
        Copies a file to another place.
    \item[mv] \hfill \\
        Moves a file or directory.
    \item[cat] \hfill \\
        Prints a file.
\end{description}

We also notice the appearance of \textbf{man} that allows the user to get all the necessary
information about a command just by typing man before.

\subsection{The console on Unix}
From next year (or right now if you want!) you'll be using
Unix compliant operating systems, in which the command line interface
is still very present and has been developed extensively in order to offer unlimited use.\\
One of the reasons why the vast majority of the servers run on this kind of OS is that when you connect to them you only have access to a TTY.\footnote{http://en.wikipedia.org/wiki/Teleprinter}\\
We will not speak in detail about all the wonders you can find in these environments
but if you're interested don't hesitate to search on the internet or ask your ACDC.

\subsection{The use in \csharp}
The management of the console in \csharp is done through\ldots the Console class!\\
Its reference is available on the \csharp bible: MSDN.\\
You can access it here:
http://msdn.microsoft.com/en-us/library/system.console\\
Your first work in this practical will be to search (not necessarily in the above page)
at least the following functions to understand how to use them:

\begin{enumerate}
    \item Console.Write
    \item Console.WriteLine
    \item Console.Read
    \item Console.ReadLine
    \item Convert.ToInt32
    \item Convert.ToFloat
\end{enumerate}

\subsection{The Main function}
When you will compile and run your program, the first and only function that will be called is \codeil{Main()}.
This function is the entry point of your program, so if you want to call your functions, you have to do it here.

\newpage
