\subsection{Exercice 4: Dans ton QCM}

Écrivez la fonction \codeil{QCM} qui pose une question à l'utilisateur et affiche:\\
- "Good job bro!" si la réponse est correcte.\\
- "You lose... The answer was X." si la réponse est fausse, en remplaçant X par
le numéro de la bonne réponse.\\
- "So counting is too hard, n00b..." si l'entrée est invalide.\\
Les Paramètres \codeil{ansX} contiennent seulement le texte de l'option, c'est à vous
d'afficher le "X) " au début de chacune.\\
Le paramètre \codeil{answer} contient le numéro de la bonne réponse.\\
Vous n'avez pas à gérer le cas où l'utilisateur n'entrerait pas un nombre.\\
PROTIP: Pour concaténer deux strings vous pouvez utiliser l'opérateur \codeil{+}.\\

Prototype:
\begin{code}
static void QCM(string question,
                string ans1,
                string ans2,
                string ans3,
                string ans4,
                int answer)
\end{code}
\newpage
Exemple:
\begin{verbatim}
> test_QCM.exe
Quelle est la différence entre un pigeon ?
1) Les deux pattes, surtout la gauche
2) Oui
3) Obiwan Kenobi
4) La réponse D
2
You lose... The answer was 1.
\end{verbatim}

Bonus: gérer les entrées incorrectes (cherchez comment utiliser \codeil{try \ldots{} catch}).
