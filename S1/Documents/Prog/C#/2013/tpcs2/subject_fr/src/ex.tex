\section{Exercices}

\subsection{Avant de commencer}
Au cours de ces exercices vous devrez souvent afficher du texte en console,
sans indication contraire tout affichage doit se terminer par un retour à la
ligne.\\
Vous remarquerez aussi que si vous lancez votre programme avec Visual Studio,
il va apparaitre et disparaitre bien trop rapidement pour que vous puissiez y
voir quoi que ce soit. Pour remédier à ce problème vous pouvez soit ajouter un
appel à \codeil{Console.Read()} à la fin de votre \codeil{Main} afin de rester ouvert en
attendant une entrée de l'utilisateur ou simplement lancer votre programme
depuis une invite de commande Windows.\\

Il est temps de rentrer dans le vif du sujet, créez un nouveau projet console
dans Visual Studio en faisant File/New/Project/Console Application avec pour
nom "tpcs02" et lancez-vous dans les exercices suivants !\\

% Ex0
\subsection{Exercice 0: Hello World}

Vous devez écrire la fonction \codeil{HelloWorld} qui affiche dans la console le texte
"Hello World!".

Prototype:
\begin{code}
public static void HelloWorld()
\end{code}

Exemple:
\begin{verbatim}
> test_helloworld.exe
Hello World!
\end{verbatim}

% Ex1
\subsection{Exercise 1: Hey ! Echo\ldots{} cho\dots{} o\dots}

Write the \codeil{Echo} function that reads a line typed by the user and prints it afterwards.

Prototype:
\begin{code}
static void Echo()
\end{code}

Example:
\begin{verbatim}
> test_echo.exe
test
test
\end{verbatim}


% Ex2
\subsection{Exercise 2: Esrever}

Write the \codeil{Reverse} function that reads a line and prints it reversed afterwards.\\
This function must be imperatively recursive.\footnote{And not recursively imperative.}\\
PROTIP: let s be a string, \codeil{s[i]} is the character at index i (beware, the indexes start at 0)
and \codeil{s.Length} is the length of s.\\
The Reverse function may call another function to do the recursion.

Prototype:
\begin{code}
static void Reverse()
\end{code}

Example:
\begin{verbatim}
> test_reverse.exe
test
tset
\end{verbatim}


% Ex3
\subsection{Exercise 3: Triforce}

Write the \codeil{Triforce} function that prints a triforce in yellow on a blue background
and changes the title to "Triforce".\\
You must also call \codeil{Console.ResetColor()} at the end to reset the console to its normal state.\\
PROTIP: these changes are not done through function calls but by modifying the \codeil{Console} properties
(look for "Console Class" on MSDN).
Also search for \codeil{ConsoleColor}.\\

Prototype:
\begin{code}
static void Triforce()
\end{code}

Example:
\begin{verbatim}
> test_triforce.exe
   /\
  /__\
 /\  /\
/__\/__\
\end{verbatim}

Bonus: Use only one \codeil{Write}, because one line is better.


% Ex4
\subsection{Exercice 4: Dans ton QCM}

Écrivez la fonction \codeil{QCM} qui pose une question à l'utilisateur et affiche:\\
- "Good job bro!" si la réponse est correcte.\\
- "You lose... The answer was X." si la réponse est fausse, en remplaçant X par
le numéro de la bonne réponse.\\
- "So counting is too hard, n00b..." si l'entrée est invalide.\\
Les Paramètres \codeil{ansX} contiennent seulement le texte de l'option, c'est à vous
d'afficher le "X) " au début de chacune.\\
Le paramètre \codeil{answer} contient le numéro de la bonne réponse.\\
Vous n'avez pas à gérer le cas où l'utilisateur n'entrerait pas un nombre.\\
PROTIP: Pour concaténer deux strings vous pouvez utiliser l'opérateur \codeil{+}.\\

Prototype:
\begin{code}
static void QCM(string question,
                string ans1,
                string ans2,
                string ans3,
                string ans4,
                int answer)
\end{code}
\newpage
Exemple:
\begin{verbatim}
> test_QCM.exe
Quelle est la différence entre un pigeon ?
1) Les deux pattes, surtout la gauche
2) Oui
3) Obiwan Kenobi
4) La réponse D
2
You lose... The answer was 1.
\end{verbatim}

Bonus: gérer les entrées incorrectes (cherchez comment utiliser \codeil{try \ldots{} catch}).


% Ex5
\subsection{Exercise 5: \_\_ \_\_\_ .\_. ... .}

Write the \codeil{Morse} function that produces the sound corresponding to the message given in input.\\
This message will be constituted of the '\_', '.' and ' ' characters only.\\
A '\_' correponds to a 450ms beep, a '.' to a 150ms beep and ' ' to a 450ms pause.\\
The sounds must have a frequency of 900Hz.\\
This function must, as usual, be recursive.\\
PROTIP: you can use \codeil{Console.Beep(f, n)} to produce a sound of frequency f Hz during n milliseconds
and \codeil{System.Threading.Thread.Sleep(n)} to pause the program for n milliseconds.\\

Prototype:
\begin{code}
static void Morse()
\end{code}

Example:
\begin{verbatim}
> test_morse.exe
.___ .____. ._ .. __ . __ . ... ._ _._. _.. _._.
bip biiip biiip… (nothing is displayed)
\end{verbatim}


% Bonus 1
\subsection{Bonus 1: Dessine-moi un poney}

Écrivez la fonction \codeil{DrawPony} qui dessine le plus joli poney en ASCII art avec
des couleurs et tout ce qui peut vous plaire.\\

Prototype:
\begin{code}
static void DrawPony()
\end{code}

Bonusception: animez votre poney !


% Bonus 2
\subsection{Bonus ultime: sl}

Écrivez la fonction \codeil{Sl} qui reproduit le comportement de la commande \textbf{sl} sour Unix.\\
PROTIP: Google est votre ami.\\
Bonusception: gérez les arguments en ligne de commande.\\
(regardez du côté de \codeil{string[] args}, le paramètre de \codeil{Main})


